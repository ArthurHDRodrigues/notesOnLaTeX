\documentclass[]{article}

%%%%%%%%%%%%%%%%%%%
% Packages/Macros %
%%%%%%%%%%%%%%%%%%%
\usepackage{amssymb,latexsym,amsmath} % Standard packages
\usepackage{bbm}      
\usepackage[brazil]{babel}
\usepackage[utf8]{inputenc}
\usepackage{ marvosym }


%  Para usar o tikz %
\usepackage{pgf,tikz,pgfplots}
\pgfplotsset{compat=1.15}
\usepackage{mathrsfs}
\usetikzlibrary{arrows}

%formatação customizada
\usepackage{indentfirst}   %indenta o primeiro paragrafo
\linespread{1.3}


%%%%%%%%%%%
% Margins %
%%%%%%%%%%%
\addtolength{\textwidth}{1.0in}
\addtolength{\textheight}{1.00in}
\addtolength{\evensidemargin}{-0.75in}
\addtolength{\oddsidemargin}{-0.75in}
\addtolength{\topmargin}{-.50in}


%%%%%%%%%%%%%%%%%%%%%%%%%%%%%%
% Theorem/Proof Environments %
%%%%%%%%%%%%%%%%%%%%%%%%%%%%%%
\newtheorem{theorem}{Teorema}
\newenvironment{proof}{\noindent{\bf Proof:}}{$\hfill \Box$ \vspace{10pt}} 
\newtheorem{proposicao}[theorem]{{\bf Proposição}}


%%%%%%%%%%%%%%%%%%%%%%%
%       Macros        %
%%%%%%%%%%%%%%%%%%%%%%%

\newcommand{\defi}{\textbf}                  %definir coisas em texto
\newcommand{\defmm}{\overset{\text{def}}{=}} %definir coisas em math mode

%       Matrizes e vetores
\newcommand{\transposta}{^\top}   %simbolo para transposta de alguma matriz
\newcommand{\um}{\mathbbm{1}}     %precisa do pacote bbm


%%%%%%%%%%%%
% Document %
%%%%%%%%%%%%
\begin{document}

\begin{titlepage}
	\begin{center}{\scshape
		Universidade de São Paulo \\
		Instituto de Matemática e Estatística}\\
		\vspace{5cm}
		{\huge
			Estilo não é mara
		}\\
		\vspace{.5cm}
		{\large Arthur Henrique Dias Rodrigues}\footnote{NºUSP: 9793951}\\
		{\footnotesize sob a orientação de}\\
		{\large José Coelho de Pina}
	\end{center}
	\vspace{3cm}
	\vfill
	\begin{center}
		Relatório \\
		MAT0148 - Introdução ao Trabalho Científico
	\end{center}

\end{titlepage}
\newpage


\title{Estilo não é mara}
\author{Arthur H. D. Rodrigues}
\maketitle
\tableofcontents
\newpage
\begin{abstract}
This document represents the output from the file ``sample.tex" once compiled using your favorite \LaTeX compiler.  This file should serve as a good example of the basic structure of a ``.tex" file as well as many of the most basic commands needed for typesetting documents involving mathematical symbols and expressions.  For more of a description on how each command works, please consult the links found on our course webpage.

\end{abstract}


\section{Imagens}
Colocar figuras no ambiente figure com caption e labelar com fig:blabla.
Não esquecer de ponto final no caption

\begin{figure}[htb]
\begin{center}
\begin{tikzpicture}[line cap=round,line join=round,>=triangle 45,x=1cm,y=1cm]
\clip(-0.47614898458867905,2.3824287310797314) rectangle (7.414798428208247,5.961490285640179);
\draw [line width=2pt] (2.3890299553441805,2.613109798564165)-- (3.3913800563064553,2.5941975325082733);
\draw [line width=2pt] (3.3913800563064553,2.5941975325082733)-- (4.213414673373465,3.168063794846413);
\draw [line width=2pt] (4.213414673373465,3.168063794846413)-- (4.541144522754616,4.115511178362278);
\draw [line width=2pt] (4.541144522754616,4.115511178362278)-- (4.2493879411141915,5.074646985104966);
\draw [line width=2pt] (4.2493879411141915,5.074646985104966)-- (3.449586026197352,5.67911393672582);
\draw [line width=2pt] (3.449586026197352,5.67911393672582)-- (2.447235925235077,5.698026202781712);
\draw [line width=2pt] (2.447235925235077,5.698026202781712)-- (1.6252013081680676,5.1241599404435725);
\draw [line width=2pt] (1.6252013081680676,5.1241599404435725)-- (1.2974714587869167,4.176712556927707);
\draw [line width=2pt] (1.2974714587869167,4.176712556927707)-- (1.58922804042734,3.217576750185019);
\draw [line width=2pt] (1.58922804042734,3.217576750185019)-- (2.3890299553441805,2.613109798564165);
\draw [line width=2pt] (2.447235925235077,5.698026202781712)-- (1.58922804042734,3.217576750185019);
\draw [line width=2pt] (2.3890299553441805,2.613109798564165)-- (4.541144522754616,4.115511178362278);
\draw [line width=2pt] (1.6252013081680676,5.1241599404435725)-- (4.2493879411141915,5.074646985104966);
\draw [line width=2pt] (3.449586026197352,5.67911393672582)-- (4.213414673373465,3.168063794846413);
\draw [line width=2pt] (3.3913800563064553,2.5941975325082733)-- (1.2974714587869167,4.176712556927707);

\begin{scriptsize}
\draw [fill=black] (2.3890299553441805,2.613109798564165) circle (2.5pt);
\draw [fill=black] (3.3913800563064553,2.5941975325082733) circle (2.5pt);
\draw [fill=black] (4.213414673373465,3.168063794846413) circle (2.5pt);
\draw [fill=black] (4.541144522754616,4.115511178362278) circle (2.5pt);
\draw [fill=black] (4.2493879411141915,5.074646985104966) circle (2.5pt);
\draw [fill=black] (3.449586026197352,5.67911393672582) circle (2.5pt);
\draw [fill=black] (2.447235925235077,5.698026202781712) circle (2.5pt);
\draw [fill=black] (1.6252013081680676,5.1241599404435725) circle (2.5pt);
\draw [fill=black] (1.2974714587869167,4.176712556927707) circle (2.5pt);
\draw [fill=black] (1.58922804042734,3.217576750185019) circle (2.5pt);
\end{scriptsize}
\end{tikzpicture}
\end{center}
\caption{Um grafo bem simétrico.}
\label{fig:grafo}
\end{figure}

Figuras de Tikz usam [line width=1pt] e circle (1.7pt)


\section{math mode}

\subsection{Denifições, provas e teoremas}

Defino a \defi{matriz laplaciana} por escrito desta forma e em math mode dessa forma:
$ L \defmm D-A$

Um teorema podem ter nome, como por exemplo.

\begin{theorem}[Binomial Theorem]
For any nonnegative integer $n$, we have
$$(1+x)^n = \sum_{i=0}^n {n \choose i} x^i$$
\end{theorem}

Proposições usam o mesmo contador.
\begin{proposicao}
Estética é uma merda
\end{proposicao}

\begin{proof}
\begin{eqnarray*}
(A\cup B)-(C-A) &=& (A\cup B) \cap (C-A)^c\\
&=& (A\cup B) \cap (C \cap A^c)^c \\
&=& (A\cup B) \cap (C^c \cup A) \\
&=& A \cup (B\cap C^c) \\
&=& A \cup (B-C)
\end{eqnarray*}
\end{proof}

Com o ambiente align posso enumerar equações e, em seguida, as referenciar.

%\begin{align}
%  A(Qe_i) & = QDQ\transposta(Qe_i) \label{eq:l0}\\
%          & = QD(Q\transposta Q)e_i \nonumber \\
%          & = QD\Id e_i \label{eq:l1}\\
%          & = QD e_i \nonumber \\
%          & = D_{i,i}(Qe_i), \label{eq:l2}
%\end{align}
onde as igualdades \eqref{eq:l0}, \eqref{eq:l1} e \eqref{eq:l2} seguem do teorema da
decomposição espectral.

\section{Símbolos}
\subsection{Variações de fontes}
{\sc Small Caps} 

{\sf Sans Serif}

\subsection{Alguns símbolos interessantes}
Segue uma lista de alguns símbolos que acho ou uteis ou só legais para usar.
\subsubsection{Uteis}
$\therefore$

\subsubsection{legais}
Sillywalk symbol \Denarius


\subsection{Pontos}
O produto de $ x_1, x_2, \dots, x_n$ é igual a $x_1 x_2 \cdots x_n$.

\end{document}